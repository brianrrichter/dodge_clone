\documentclass[12pt,a4paper]{article}

\usepackage[brazil]{babel}
\usepackage[utf8]{inputenc}
\usepackage[T1]{fontenc}
\title{Dodge Clone}
\author{Brian R. R.}
\date{\today}

\begin{document}
\begin{titlepage}
\maketitle
\end{titlepage}

\begin{abstract}
O presente documento tem por objetivo descrever o jogo Dogge Clone.
\end{abstract}

\tableofcontents{}

\section{Análise do jogo}

%   Introduce the game. Present information on why this game will be fun, the 
% purpose of the game, what the player does, and so on. This is meant to be
% a quick analysis of the game and what you can expect from it. Shouldn’t
% be more than 1-2 paragraphs.

Este é um jogo de sobrevivência, o jogador deve ficar o máximo de tempo vivo.
Quanto mais tempo ele fica vivo, maior é o score.

O jogo se torna divertido pela competição que gera entre os jogadores.
Se o jogador ficar entre as 10 primeiras posições, ele pode colocar o nome dele
na tela de Score.

\section{Declaração da missão}

%   In 1-2 sentences, explain the game as if you were pitching it to potential
% players. This should be very intriguing. It typically includes the title,
% genre, platform, and brief idea of what the player does or has to overcome.

\section{Gênero}

% List or describe the game’s genre/genres

\section{Plataformas}

% List or describe the platforms the game will be made for.

\section{Público alvo}

%   Provide information on the audience the game is targeted to. Add details and
% information on the intended audience such as their habits, behaviors, likes, 
% and dislikes. Are you targeting your game to a specific age group or perhaps 
% people that enjoy certain genres? Is your intended audience from specific 
% communities or will their locale play a role?

\section{Enredo}

\section{Jogabilidade}

\subsection{Visão geral da jogabilidade}

\subsection{Experiência do jogador}

\subsection{Diretrizes de jogabilidade}

\subsection{Objetivos do jogo e recompensas}

\subsection{Mecânica da jogabilidade}

\subsection{Design de nível}

\section{Esquema de controle}

\section{Estética do jogo e interface de usuário}

\section{Agenda e tarefas}

\section{extras}

O template para este documento foi baseado no template baixado em: https://vitalzigns.itch.io/gdd

\end{document}
