\documentclass[12pt,a4paper]{article}

\usepackage[brazil]{babel}
\usepackage[utf8]{inputenc}
\usepackage[T1]{fontenc}
\title{Dodge Clone}
\author{Brian R. R.}
\date{\today}

\begin{document}
\begin{titlepage}
\maketitle
\end{titlepage}

\begin{abstract}
O presente documento tem por objetivo descrever o jogo Dogge Clone.
\end{abstract}

\tableofcontents{}

\section{Análise do jogo}

%   Introduce the game. Present information on why this game will be fun, the 
% purpose of the game, what the player does, and so on. This is meant to be
% a quick analysis of the game and what you can expect from it. Shouldn’t
% be more than 1-2 paragraphs.

Este é um jogo de sobrevivência, o jogador deve ficar o máximo de tempo vivo.
Quanto mais tempo ele fica vivo, maior é o score.

O jogo se torna divertido pela competição que gera entre os jogadores.
Se o jogador ficar entre as 10 primeiras posições, ele pode colocar o nome dele
na tela de Score.

\section{Declaração da missão}

%   In 1-2 sentences, explain the game as if you were pitching it to potential
% players. This should be very intriguing. It typically includes the title,
% genre, platform, and brief idea of what the player does or has to overcome.

Teste suas habilidades e veja quanto tempo você consegue se manter de pé!

No Dodge Clone o jogador se depara com um ataque de criaturas vindo por todos
os lados e deve desviar para sobreviver.

Compare o seu desempempenho com seus amigos ou com o seu próprio. Só os
melhores deixarão suas marcas na história.

\section{Gênero}

% List or describe the game’s genre/genres

\section{Plataformas}

% List or describe the platforms the game will be made for.

\section{Público alvo}

%   Provide information on the audience the game is targeted to. Add details and
% information on the intended audience such as their habits, behaviors, likes, 
% and dislikes. Are you targeting your game to a specific age group or perhaps 
% people that enjoy certain genres? Is your intended audience from specific 
% communities or will their locale play a role?

\section{Enredo e personagens}

%   This is where you present a story synopsis, and discuss how the story will 
% unfold as the player makes his or her way through the game. Include information 
% on the key characters in the game, including descriptions, abilities, 
% characteristics, how they fit into the story, how they affect gamplay, what the 
% player will learn from them, etc.

% Character
%   Character Name
%   Character Image

% Description
%   Describe the character. It is a playable character or NPC. How does this 
%       character fit into the story, etc.

% Characteristics
%   Describe the character’s abilities, personality and so forth.

% Misc. Info
%   Present any other notes about the character.

\section{Jogabilidade}

\subsection{Visão geral da jogabilidade}

%   Include information on the game genre and how it is different, similar, or a
% hybrid of existing genres. Discuss what platform the game will be on, if it is
% going to be on multiple platforms discuss ways the game will be modified for
% each platform. Also, provide a general overview of the game modes available in
% single player and multiplayer. Also, list the Key Gameplay Features (selling
% features) of the game.

\subsection{Experiência do jogador}

%   Provide a general overview of how the player experiences the game. Walk them
% through the screens they will see, what the level looks like and what their
% character can do. Give them a brief idea of objectives & hazards they will
% face.  This should be in a second-person point of view using the word “you” to
% tell a story to the audience (players).

\subsection{Diretrizes de jogabilidade}

%   This is a set of guidelines that the game must adhere to throughout the
% development process. These include rules for what is allowed and not allowed in
% the game. For instance, if you are creating a game for children, you will want
% to define guidelines for the level of violence presented in the game, what
% language can be used, and so on.

\subsection{Objetivos do jogo e recompensas}

%   This is where you present more details on how the gameplay will motivate the
% player to progress through the game. Discuss rewards and penalties and the
% difficulty level. You can use the table below to help break down objectives and
% rewards.

% Rewards
%   List ways of how the player is rewarded and when.

% Penalties
%   Discuss things that hinder the player on progressing

% Difficulty Levels
%   Discuss the difficulty levels within the game

\subsection{Mecânica da jogabilidade}

%   This is the where you start getting more specific on how some of the systems in
% the game will work. This includes how characters move in the game, what
% gameplay actions are available, item inventory and attributes, and how the game
% progresses from level to level.

% Character Attributes
%   Character
%     Name of character
%   Movement Abilities / Actions Available
%     List the characters abilities & how the player can perform them

% Game Modes
%   Game Mode / Difficulty Name
%   Describe the objectives, hazards in the game mode. And discuss how the player progresses from level to level

% Scoring System
%   Points/Coins/Stars/Grades/Etc.
%     List the scoring attribute
%   How it’s Awarded & Benefits
%     Describe how the player obtains this and the benefits. For instance, does getting more points unlock a special level.

\subsection{Design de nível}

%   Discuss the levels. How many levels will the game have, what will be included
% in each level. Include overall look and feel, hazards the level presents,
% difficulty, objectives, etc.

% Level name and/or pic of it IMAGE
% List or describe the level’s look, difficulty, hazards, and objectives.

\section{Esquema de controle}

%   Describe the control setup for the game. Does your game use touch input, a
% controller, or mouse & keyboard? Discuss the functionality of each
% button/touch. It may help to insert a diagram/pic to help explain the actions.

% Button/ Touch Input  |   Action it Performs
% List the button      |   Describe what functionality the button press has within the game.

\section{Estética do jogo e interface de usuário}

%   Discuss the design techniques to be used. Describe the look & shape of the
% characters, environment and pathways. Will the game look realistic or have some
% other art style. Discuss what type of theme the game will have & what type of
% emotional impact you are hoping players experience. Discuss how the player’s
% gestures/interactivity has an affect on the visual experience. 

%   Present a general overview of the UI. How will the buttons be laid out, how
% will the HUD work, how does the menu system function, and so on. It is a good
% idea to insert photos, diagrams or concept art to help explain the UI.

\section{Agenda e tarefas}

%   List the tasks that need to be completed along with the basic timeline to
% complete them by. The task list can be as detailed as you like to fit your
% studio’s needs. The table below can be substituted for the excel file. This
% table is a great start but the tasks should be much more detailed.

% table of tasks

\section{extras}

O template para este documento foi baseado no template baixado em: https://vitalzigns.itch.io/gdd

\end{document}

% break lines at 80: g q motion
% help: gq
% help: gw
